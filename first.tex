\documentclass[twocolumn,10pt]{article}


\title{Energy Depletion Attacks on Wireless Sensor Networks : A Survey}

\author{FARZANA T\\12MCS1004\\Guided by: Mrs. Aswathy Babu C.A(Asst.Professor CSE Dept)}

\usepackage{epsfig}
\begin{document}
	
\maketitle
	
{\bf Keywords:}
Wireless Senser Network, Denial of Service,multi-path routing, opportunistic routing, energy efficiency

\abstract{} Deployment of sensor network in hostile environment makes it mainly vulnerable to battery drainage attacks because it is impossible to recharge or replace the battery power of sensor nodes.Most of the research on this topic is revolved around security  solutions using the layered approach.Here analysis is mainly focused on minimization of energy consumption at MAC layer and routing layer.Some of such innovations are analysed here.

\section{\label{reflabel} Introduction}

A wireless sensor network consists of spatially distributed  autonomous sensors to cooperatively monitor physical or  environmental conditions such as pressure, temperature,  sound, vibration motion or pollutants. WSN is used to locate  not only the objects whose area of location is known but also  the objects whose location is anticipated to be around a  certain domain. Each node in a sensor network is typically  equipped with a radio receiver, a small micro controller,  energy source usually a battery. Sensor networks can be used  for target tracking, system control and chemical and  biological detection. Sensor networks are typically  characterized by restricted power supplies, low bandwidth,  small memory size and limited energy.  This leads to a very  demanding environment to provide security.

Sensor networks  can  be pushed to resource consumption attack. This  means enemies would send data to drain a node battery and  reduce network bandwidth. Sensor network is typically the cluster based  and has irregular topology. Clusters are interconnected to the  main base station. Each cluster contains a cluster head  responsible for routing data from its corresponding cluster to  a base station. Sensor networks often have one or more points  of centralized control called base station. The wireless sensor  node is equipped with a limited power source such as battery,  sensor unit, processing unit, storage unit and wireless radio  transceiver; these units communicate each other. A base  station is typically a gateway to another network, a powerful  data processing or storage center or an access point for  human interface; communicating nodes are normally linked  by a wireless medium such as radio. 




\section{Literature Survey}

Most of the research on this topic is revolved around security  solutions using the layered approach. In layered approach the protocol  stacks consists of the physical layer, data link layer, network  layer, transport layer and application layer.These five layers  and the three planes, i.e., the power management plane,  mobility management plane and task management plane  jointly forms the wireless layered architecture.

Researches are always being conducted to improve the energy efficiency of the 
wireless Sensor Networks. Some of the approaches are described. 

Michael Brownfield[1] discussed the energy resource vulnerabilities at MAC level and proposed a new G-MAC protocol to control the sleep awake pattern of sensor nodes.This scheme performs well in all traffic situations but deals only with MAC layer depletion attack.

Jing Deng, Richard Han, Shivakanth mishra[2] proposed an Intrusion tolerant routing protocol for WSN. INSENS constructs a forwarding table at each node to facilitate communication between sensor nodes and base station.It also provides multi path routing and minimize the communication,storage and computation requirements of sensor node at the expense of increased requirements at base station.


Fatma Bouabdullah, Nizar Bouabdullah,Raouf Bouabdullah [3] proposed a cross layer strategy that considers routing and MAC layers jointly. At routing level they proposed that sending data through multiple paths instead of using a single path, at MAC level limits the retransmission over each wireless links, but this scheme does not considers any attack. 

Xufei Mao[4]. focused on selecting and prioritizing forwarder list to minimize energy consumption by all nodes but this method does not consider any attack at routing level.

Tapaliana Bhattasali[5] proposed an  frame work based on distributive collaborative mechanism for detecting sleep deprivation attack  increased energy efficiency but does not considers routing layer  

E.Y Vasserman & N. Hopper[6] proposed a new method for resource depletion attack at routing layer, which permanently disable networks by quickly draining nodes battery power.

\section{Conclusion}

The absence of infrastructure in WSN makes  it difficult to detect security threats.Therefore security mechanism have to be designed with efficient resource utilisation, especially power.Here analyses the energy resource vulnerabilities of WSN at MAC level.The vampire attack discussed in this survey explores the energy consumption attack at routing layer.

\begin{thebibliography}{99}

 \bibitem{key1}Michael Brownfield,Yatharth Gupta,
 \newblock "Wireless Sensor Network Denial of Sleep Attack",
 \newblock \emph{Proceedings of 2005 IEEE workshop on information assurance,June 2005}.

 
  \bibitem{key1}Jing Deng, Richard Han, Shivakanth mishra,
 \newblock "INSENS: Intrusion-Tolerant routing in Wireless Sensor Networks",
 \newblock \emph{University of Colorado,Department of computer science Technical report,June 2006 }.

\bibitem{key1}Fatma Bouabdullah, Nizar Bouabdullah,Raouf Bouabdullah 
 \newblock "Cross-layer Design for Energy Conservation in  Wireless Sensor Networks",
 \newblock \emph{IEEE GLOBECOM 2008,New Orleans,USA,December 2008}.

 \bibitem{key1}Xufei Mao,Shaojie Tang, Xiahua Xu,
 \newblock "Energy efficient Oppurtunistic Routing in Wireless Sensor Networks",
 \newblock \emph{IEEE transactions on parellel and distributed systems, VOL. 12, NO. 2, February 2011 }

\bibitem{key1}Tapaliana Bhattasali,Rituparna Chaki,Sugata Sanyal
 \newblock "Sleep Deprivation Attack Detection in Wireless Sensor Networks",
 \newblock \emph{International journal of computer applications(0975-8887)vol 40- No: 15,February 2012 }
 
 \bibitem{key1}Eugene Y. Vasserman, Nicholas Hopper,
 \newblock " Vampire Attacks: Draining Life from Wireless Ad Hoc Sensor Networks",
 \newblock \emph{IEEE transactions on mobile computing, VOL. 12, NO. 2, February 2013 }
 
  \bibitem{key1}Yazeed Al-Obaisat,Robin Braun,
 \newblock "On Wireless Sensor Networks: Architectures,Protocols,Applications and Management",
 \newblock \emph{Institute of Information and Communication Technologies,May 2004 }
   
\end{thebibliography}

\end{document}
