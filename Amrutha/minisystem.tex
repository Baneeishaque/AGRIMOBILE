\chapter{System Analysis}
\label{System Analysis}

	System analysis is the process of gathering and interpreting facts, diagnosing and using this information to recommended improvements to the system. The objectives of the system analysis phase are the establishment of the requirements for the system to be acquired, developed and installed. System analysis is for finding out what happens in the existing system deciding on what changes and features are required and defining exactly what the proposed system must be Analysis specifies what the system should do.

\section{Existing System}

\label{Existing System}

	Currently for parental control tasks in computational environments, using security system such as IPCam, false alarm  etc. The environments are fully networked for various purposes and every user had network access according to his role. There also some earlier solutions for remote access such as Telnet, Symantec�s PC anywhere, etc.

\subsection{Disadvantages}
\label{Disadvantages}

	Utilities like Telnet and remote control programs like Symantec's PC anywhere let you access remote systems. But using these solutions, the users are not able to access the desktop of the remote machine & the user will never get the feeling that they are working in the remote machine. There is no provision to shut down or reboot remote system, send messages to the server & use the processor of the remote machine directly. These systems are highly expensive and renew every year.
	


\section{Proposed System}
\label{Proposed System}

	PC Control is a client/server software package allowing remote network access to PC�s graphical desktop. This software enables you to get a view of the remote machine desktop and thus control it with your android touch pointer. It can be used to perform remote system control and administration tasks in Windows computational environments with assorted network capabilities.

	The mobile user will register by providing his username & password through android application. The server access will be provided only for the registered users. There will be an option to send messages to the server machine; these messages will notified on the server screen using server foreground application. In addition to the control through desktop; there will be shortcuts (i.e. single click actions) for power options.

\subsection{Features}
\label{Features}
\begin{itemize}
\item Communication based on TCP/IP.
\item Accessing the PC using a smart phone.
\item Select a PC from the network.
\item Controlling applications in PC.
\item Sending messages to the users who use the server.
\item Controlling PCs power functions like Shut down, Sleep & Hibernate.
\end{itemize}

